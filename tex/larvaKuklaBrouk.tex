% !TEX program = xelatex
\documentclass[a4paper, 12pt]{article}

\setlength{\hoffset}{-1.6cm} 
\setlength{\voffset}{-1.6cm}
\setlength{\textheight}{24.0cm} 
\setlength{\textwidth}{17cm}

\usepackage{float} %použití [H] u figur, umístění na přesné místo
\usepackage{amssymb}
\usepackage[intlimits]{amsmath}
\usepackage{xltxtra,polyglossia}
\usepackage{tikz} % Export grafů z RStudia
\usepackage{placeins} % Umožňuje používat \FloatBarrier
\usepackage[super,square]{natbib} % Citace s horním indexem a v hranatých závorkách
\usepackage[nottoc]{tocbibind} % Aby se texal "Seznam použité literatury" místo "Reference"
\usepackage{color,graphicx,graphics}
\usepackage{booktabs,paralist,url}
\usepackage{pdfpages}
\usepackage{fancyhdr}
%\usepackage{siunitx} %$$\SI{666\pm 5 e3}{cm^{-1}}$$

\setmainlanguage{czech}
\renewcommand{\d}[1]{\ensuremath{\operatorname{d}\!{#1}}} %pro psaní \d jako derivace
\newcommand{\U}[1]{{\ \rm #1}} % sazba jednotek
\newcommand{\E}[1]{\cdot 10^{#1}} % sazba exponentu (10^exponent)
\newcommand{\EU}[2]{\E{#1}\U{#2}} % sazba exponentu a jednotek jednotek


\usepackage{newfloat} %pro zadefinování graph
\DeclareFloatingEnvironment{graph} %zadefinuje graph
\addto\captionsczech{ %vytvoří 2 typy figur
  \renewcommand{\graphname}{Graf} %\begin{graph}
  \renewcommand{\figurename}{Obrázek}%\begin{figure}
}
%\bibliography{literatura}
%\input{dependencies/slovnik} % explicitní slovník slov, která se mají zalamovat na konci řádku ( nejnovější verze vždy na https://gitlab.com/skolar/slovnik.git )

%\usepackage{fkssugar}
%\usepackage{longtable}
%\usepackage{python}

%pro pdfTEX
%\usepackage[utf8]{inputenc}
%\usepackage[T1]{fontenc}
%\usepackage{lmodern}
%\usepackage[czech]{babel}


% \newcommand{\AUTOR}{}

% \usepackage{fancyhdr}
% \pagestyle{fancy}
% \lhead{}
% \rhead{\AUTOR}

\begin{document}
\section{Moučný červ (Tribolium)}
Vývoj tohoto červa je postupně: vajíčko $\rightarrow$ larva $L$ (2 týdny)$\rightarrow$ kukla $P$ (2 týdny)$\rightarrow$ brouk $A$. Vývoj jednotlivých fází lze popsat vztahy\\
\begin{equation}
    \begin{split}
\text{Larva: }& L_{n+1}=b A_n \\
\text{Kukla: }& P_{n+1}=L_n(1-\mu_l) \\
\text{Brouk: }& A_{n+1}=P_n(1-\mu_p)+A_n(1-\mu_a), \\
    \end{split}
\end{equation}
kde $\mu_l$ je úmrtnost larev, $\mu_p$ úmrtnost kukel a $\mu_a$ úmrtnost brouků.


Pokud zahrneme kanibalismus, který se u těchto druhů často vyskytuje, dostaneme: \\
\begin{equation}
\begin{split}
\text{požírání larev: }&L_{n+1}=b A_n e^{-c_{la}A_n-c_{ll} L_n} \\
\text{požírání kukel: }&P_{n+1}=L_n(1-\mu_l) \\
\text{požírání brouků: }& A_{n+1}=P_n(1-\mu_p)e^{-c_{pa} A_n}+A_n(1-\mu_a), \\
\end{split}
\end{equation}
kde koeficienty byly zjištěny experimentálně jako
\begin{equation}
\begin{split}
c_{la}& = 0.009 \text{ (A požírá L)}\\
c_{ll}& = 0.012 \text{ (L požírá L)}\\
c_{pa}& = 0.004 \text{ (A požírá P)}\\
\mu_l& = 0.267 \\
\mu_p& = 0\\
\mu_a& = 0.0036 \text{ (základní úmrtnost)} \\
b& =7.48 \text{ počet nových larev na 1 dospělého brouka za jednotku času, což je 14 dní)}\\
\end{split}
\end{equation}

\subsection{Nalezení největšího Ljapuova exponentu}
Nalezení největšího Ljapuova experimentu pro $\mu_a\in[0,1]$ provedeme metodou............

.................
Fázový prostor $x=P_n$, y=$L_n$, $z=A_n$

První bifurkace je při $\mu_a=0.1$, zde se perioda $1$ mění na periodu $2$. Druhá bifurkace je v bodě $\mu_a=0.6$, perioda $2$ se mění na $1$. ke třetí bifurkaci dochází při $\mu_a=0.954$, kde vzniká chaotické chování.

projekt: 

\section{Po Thanosově lusknutí}
Předpokládejme, že populace blouků je statisticky natolik veliká, že dojde k vyhlazení poloviny populace.
Tento krok lze zapsal, jako
\begin{equation}
  \begin{split}
\text{Larva: }& L_{n+1}=\frac{L_n}{2} \\
\text{Kukla: }& P_{n+1}=\frac{P_n}{2} \\
\text{Brouk: }& A_{n+1}=\frac{A_n}{2} \\
  \end{split}
\end{equation}


---------------------------------------------------------------------\\
\today, Daniel Rod, Michal Grňo, Jan Střeleček

\end{document}

